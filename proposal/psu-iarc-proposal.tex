\documentclass{article}

\begin{document}

\begin{center}
{\Large{\bf Portland State Aerial Robotics Team}}\\*[3mm]
{\bf Portland State University designs and builds an autonomous quad-rotor flying vehicle for international competition} \\*[3mm]

Spencer Krum, Patrick Bledsoe, Gregory Haynes





Mentors: Jeff Douhtey, Graduate Student, Physics
\\Dr. Erik S\`anchez, Physics, Dr. Bart Massey, Computer Science


\end{center}


\section{Abstract} This is a  proposal to fund the Portland State University Robotics Team to compete in the 2011 International Aerial Robotics Competition (IARC) hosted by the Association for Unmanned Vehicle Systems International (AUVSI). The team is comprised of students studying mechanical engineering,
 chemistry, computer science, and physics who will build an autonomous helicopter to compete in August.  The strengths of the 2011 team are computer programming and experience from previous years. The combination of hard work 
from the team and support from you gives Portland State a realistic chance of winning the International Aerial Robotics Competition.  


\section{Results from Prior Competitions}

\noindent
{\bf PSU-ROV 2010}
{\it Total UROV cost: \$2496.06}

In June of 2010, the Portland State Robotics Team completed an underwater remotely operated vehicle (UROV) to compete in the Marine Advancement for Technology Education (MATE) Center's annual international competition. Out of over four hundred applicants for the combined competition classes, the Portland Sate University ROV team passed the local qualification round and went on to participate in the international competition. 
Five students and two mentors went to Hilo, Hawaii to compete. The 2010 ROV received 70/300 mission points and 216/500 total points, 
ranking 18th out of 26 international teams. The team learned and demonstrated the ability to produce a working machine under budget and time constraints.

\ \\
\noindent
{\bf PSU-ROV 2009}
{\it Total UROV cost: \$481.10}

In 2009, the Portland State University ROV team sent three students and one mentor to Boston, Massachusetts, where the underwater vehicle did not pass the safety inspection due to unforeseen electrical difficulties. The robot was unable to participate in the competition. The team received 0/300 mission points and 80.67/500 for the total score(mostly for the technical report), ranking 28th. 


\section{Introduction}

Autonomous flying vehicles are on the cutting edge of engineering. The IARC is specifically designed so that no existing craft can complete the mission. Because no team could complete the 2010 mission, no new tasks or alterations have been added for the 2011 mission. What the team needs in terms of departmental support is funding and a place to work. A budget outline appears at the end of this document. The team needs a place to build and test the robot. In past years the team used the SB1 201 lab space, but the construction of the robot was disruptive to the classes and tools were stolen. 

\section{Mission}

The craft will have to identify and travel through a one meter square window. There will be a camera with a blue light watching the door. When the light is on, the camera will be off and the craft will have a short time period to enter the building. Once inside, the robot must read signs on the walls written in Arabic, and navigate to a particular office. Next, the robot needs to identify and pick up a USB Flash drive and replace it with an identical one. Finally, the craft needs to escape without being detected. The total time allowed is ten minutes. 
\section{Design}

The flying robot will be a quad-rotor under computer control. The robot will be in radio communication with a remote computer that will handle the complex navigational computation. A laser, a webcam, and trigonometry will be used to build a map of forward distances. Infrared light emitting diodes and photo-sensors will be used for proximity detection. Accelerometers, gyroscopes, and magnetometers will be used to ensure level flight. 

\section{Time Line}

The final competition will be in August 2011. A first step is to purchase an off-the-shelf quad rotor for testing purposes. This will enable the software and hardware teams to work independently, because otherwise software would be waiting for hardware to build a device and for electrical to make it fly. 
\section{Budget Outline}

\begin{tabular}[t]{lr}
\it Off the shelf quad-rotor:           &\$399.99\\
\it Microcomputers/Sensors/Electronics: &\$400.00\\
\it Miscellaneous:                       &\$200.00\\\hline
\it Total:                              &\$999.99\\
\end{tabular}

\section{Conclusion}

Our team is ambitious and capable. While there are only three listed authors there are ten more students involved with the team. Most of the expenses in our projected budget have to do with the competition and not the development of the craft. The bottom line is that a team of undergraduates can build an autonomous aerial robot capable of negotiating hallways for a little less than a thousand dollars. Whether or not we need the additional funds for the competition can be assessed in the Spring or Summer. This is one of the most interesting projects at Portland State right now and it is all the more impressive because it is being taken up by undergraduates. 

\end{document}
