\documentclass{article}

\begin{document}

\begin{center}
{\Large{\bf Portland State Aerial Robotics Team}}\\*[3mm]
{\bf Portland State Univeristy designs and builds an autonomous quad-rotor flying vehicle for international competiton} \\*[3mm]

Spencer O. Krum, Patrick Bledsoe, Gregory Haynes

\end{center}


This is a  proposal to fund the Portland State University Robotics Team to compete in the  International Aerial Robotics Competition (IARC) hosted by the Association for Unmanned Vehicle Systems International (AUVSI). A team of undergraduates studying mechanical engineering,
 chemistry, computer science, and physics will build an autonomous robot to compete in August of 2011. The effort is sponsored by Portland State University, 
the PSU Physics Department, local businesses and companies, and private individuals.  The strengths of the 2011 team are computer programming and experience from previous years. The combination of hard work 
from the team and support from you gives Portland State a realistic chance of winning the International Aerial Robotics Competition.  


\centerline{\bf Results from Prior Competitions}

\noindent
{\bf PSU-ROV 2010}
{\it Total UROV cost: \$2496.06}

In 2010, the Portland State Robotics team built a robot for the Marine Advanced Technology Education Center's International Remote Operated Vehicle Competition. Out of over 400 applicants for the [combined competition classes], the Portland Sate University ROV team passed the local qualification round and went on to participate in the International competition. 
Five students and two mentors went to Hilo, Hawaii to [competed successfully]. The 2010 ROV received 70/300 mission points and 216/500 total points, 
ranking 18th out of 26 international teams.

\ \\
\noindent
{\bf PSU-ROV 2009}
{\it Total UROV cost: \$481.10}

In 2009, the Portland State University ROV team sent three students and one mentor to Boston, Massachusetts, where the robot did not pass the safety inspection 
due to unforeseen electrical difficulties.  The robot received 0/300 mission points and 80.67/500 for the total score, ranking 28th. 


\section{Introduction}

Autonomous flying vehicles are on the cutting edge of engineering. The IARC competion is specifically designed so that no existing craft can complete the mission. If no team can complete the mission, the misson stands for another year until a team completes it. The 2010 mission was not completed and so will be repeated in 2011. There is a \$1,000 entry fee and a \$20,000 grand prize in 2011. The Portland State team is a vetran of the MATE competion and strong in computer programmers. What the team needs in terms of departmental support is funding and a place to work. The cost of construction of the flyer and travel is estimated to be about \$8,000. The place to work is needed because in past years the team has used lab space where non-team member students couldn't keep their hands to themselves. Tools were stolen and setups broken. 

\section{Mission}

The craft will have to identify and travel through a 1m square window. There will be a camera with a blue light watching the door. When the light is on, the camera will be off and the craft will have a short window in which to enter the building. Once inside, the robot must read signs on the walls(in Arabic, no less) and navigate to a particular office. Next, the robot needs to identity and pick up a USB Flash drive and replace it with an different one. Finally the craft needs to escape, undetected. The total time allowed is ten minutes or less. 
\section{Design}

The flying robot will be a quad-rotor under computer control. The robot will be in radio communication with a remote computer that will handle the complex navigational computation. A laser and a webcam can be used to build a map of forward distances. 

\section{Time Line}
\section{Management Plan}

\end{document}
