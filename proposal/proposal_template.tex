\documentclass{proposalnsf}
\usepackage{epsfig}

% NSF proposal generation template style file.
% based on latex stylefiles  written by Stefan Llewellyn Smith and
% Sarah Gille, with contributions from other collaborators.

\newcommand{\jas}{{\it J. Atmos. Sci.}}
\newcommand{\jpo}{{\it J. Phys. Oceanogr.}}
\newcommand{\JPO}{{\it J. Phys. Oceanogr.}}
\newcommand{\jfm}{{\it J. Fluid Mech.}}
\newcommand{\jgr}{{\it J. Geophys. Res.}}
\newcommand{\JGR}{{\it J. Geophys. Res.}}
\newcommand{\jmr}{{\it J. Mar. Res.}}
\newcommand{\arfm}{{\it Ann. Rev. Fluid Mech.}}
\newcommand{\dsr}{{\it Deep-Sea Res.}}
\newcommand{\dao}{{\it Dyn. Atmos. Oceans}}
\newcommand{\jam}{{\it Journal of Applied Meteorology}}
\newcommand{\phfl}{{\it Phys. Fluids}}
\newcommand{\phfla}{{\it Phys. Fluids A}}
\newcommand{\PhilTrans}{{\it Philosophical Transactions of the Royal Society, 
London}}
\newcommand{\gafd}{{\it Geophys. Astrophys. Fluid Dyn.}}
\newcommand{\gfd}{{\it Geophys. Fluid Dyn.}}
\newcommand{\PCE}   {{\it Physics and Chemistry of the Earth}}
\newcommand{\PRL}   {{\it Physical Review Letters}}

\newcommand{\ProgOc}{{\it Prog. Oceanography}}
\newcommand{\WHOITR}{Woods Hole Oceanographic Institution Technical Report, WHOI-}
\newcommand{\degrees}{$\!\!$\char23$\!$}
\DeclareFontFamily{OT1}{psyr}{}
\DeclareFontShape{OT1}{psyr}{m}{n}{<-> psyr}{}
\def\times{{\fontfamily{psyr}\selectfont\char180}}


\renewcommand{\refname}{\centerline{References cited}}

% this handles hanging indents for publications
\def\rrr#1\\{\par
\medskip\hbox{\vbox{\parindent=2em\hsize=6.12in
\hangindent=4em\hangafter=1#1}}}

\def\baselinestretch{1}

\begin{document}

\begin{center}
{\Large{\bf PSU-UROV 2011}}\\*[3mm]
{\bf Portland State Univeristy enters into the annual MATE international underwater remote operated vehicle competition} \\*[3mm]

Patrick Bledsoe, Gregory Haynes, Spencer O. Krum, Devin Quirozoliver, Kristine Summerfield,
 Allan Dunham, Dan Colish, Jeff Doughty, Dr. Erik Sanchez

\end{center}


This is a  proposal to fund the Portland State University Robotics Team to compete in the  International Aerial Robotics Competition (IARC) hosted by the Association for Unmanned Vehicle Systems International (AUVSI). A team of undergraduates studying mechanical engineering,
 chemistry, computer science, and physics will build an autonomous robot to compete in [August of 2011]. The effort is sponsored by Portland State University, 
the PSU Physics Department, local businesses and companies, and private individuals.  The strengths of the 2011 team are computer programming and experience from previous years. The combination of hard work 
from the team and support from you gives Portland State a realistic chance of winning the International Aerial Robotics Competition.  



\ \\
\noindent
{\bf Intellectual Merit:}
The design, construction, and operation of remote operated vehicles, and in particular underwater remote operated vehicles, is interesting
because these vehicles are a gateway to an ever expanding frontier. Each new development in the technology behind these devices enables 
researchers to probe deeper into the ocean or further into caves and even underwater rescue and repair. ROVs and UROVs are ubiquitous becasue 
of their utility, whether a team is looking at the bottom of a boat as a part of a safety check or probing a black smoker under the pacific, 
they are doing it with an ROV because it's faster, cheaper, and safer to do so. UROVs can go places where manned submarines cannot, sometimes 
simply because the cost of deploying a manned submarine is too high.  
\ \\

\noindent
{\bf Broader Impacts:}
This project is a learning experience for the team members and a chance to win some [accolades] for Portland State. [The team members learn 
a variety of interrelated skills through this project. Students will learn the realities of deadlines, interpersonal conflicts, and budgeting, albeit 
sometimes the hard way. There is a long list of technical skills that will be developed, ranging from soldering and machining components to deploying 
multi-processor closed-loop control systems.] At the competition, the team will see other teams' solutions to the same problems and will be
 given the opportunity to network with leaders from both industry and academia. 



\centerline{\bf Results from Prior Competitions}

\noindent
{\bf PSU-ROV 2010}
{\it Total UROV cost: \$2496.06}

In 2010, the Portland State Robotics team built a robot for the Marine Advanced Technology Education Center's International Remote Operated Vehicle Competition. Out of over 400 applicants for the [combined competition classes], the Portland Sate University ROV team passed the local qualification round and went on to participate in the International competition. 
Five students and two mentors went to Hilo, Hawaii to [competed successfully]. The 2010 ROV received 70/300 mission points and 216/500 total points, 
ranking 18th out of 26 international teams.

\ \\
\noindent
{\bf PSU-ROV 2009}
{\it Total UROV cost: \$481.10}

In 2009, the Portland State University ROV team sent three students and one mentor to Boston, Massachusetts, where the robot did not pass the safety inspection 
due to unforeseen electrical difficulties.  The robot received 0/300 mission points and 80.67/500 for the total score, ranking 28th. 


\ \\
\noindent{\Large \bf PROJECT DESCRIPTION}

\section{Introduction}

Each year the MATE center hosts an underwater remote operated vehicle competition. There are two classes, one targeted at high schools and 
the other targeted at colleges and universities. The latter is called the Explorer class and is the class that PSU-UROV competes in. MATE 
publishes a list of missions and teams build UROVs specifically to complete these missions. MATE has not yet published the missions for 2011 
but some components are standard requirements (i.e. an arm). The competition will take place in a pool for controlled conditions; in the 2011 
season this pool will be NASA's Neutral Buoyancy Lab at the Johnson Space Center in Houston, TX. The UROV and all its control equipment 
must be powered from a supplied 48 volt DC power supply rated up to 40 amps provided by MATE. The team will be racing the clock, 
5 minutes to set up, 15 minutes in pool for missions, and a 5 minute clean up. The operators will not be allowed to see the pool during the mission run; all 
control of the UROV must be done through the sensors the team has installed in the UROV. The UROV and all topside control equipment must be 
powered from a MATE-supplied DC power supply that in standard for all competitors. 
\section{Mission}

MATE has not yet published the 2011 Mission objectives. Here is what is known:
\begin{itemize}
\item The 2010 mission objectives were published in late November, 2009
\item Explorer class qualifications will take place in May of 2011.
\item The mission will take place in June of 2011.
\item The Competition will be in Houston, in the pool NASA uses to train astronauts to work in zero g, ~40ft deep
\end{itemize}
It's really not a lot of information. The team is therefore working on things that aren't mission specific such as thrusters, gyroscopes, control loops, sponsorship, etc.
However, this years Challenge will be operating in the pressure at this ~40ft depth.


\section{Design}

The Portland State UROV 2011 will be a direct descendant of the two previous years' UROVs. Some techniques are being reused or modified. The tether from 2010 consisted of low gauge copper
 wire and ethernet for communications. This years' is likely to be similar copper wire but the ethernet will be replaced with serial data for commands and a coaxial cable for video. If
 funds and sponsors permit, the UROV will have upgraded thrusters of the same type used in commercial applications. Two Teensy 8bit microcontrollers will be swapped for one ARM7 32bit processor.
8 single direction thrusters will be changed to six bi-directional thrusters and a clever placement of the same will give the 2011 craft control in two more dimensions. 
Perhaps the largest change is in control scheme. The 2009 and 2010 crafts were controlled directly by their human pilots. The 2011 craft will have a closed loop control system that will automatically compensate for changes in center of mass and for prevailing currents. The pilot will give instructions and the UROV will carry them out, no compensation by the human required. 


\section{Time Line}

Meeting Agenda List (assumes that planned meetings start when the
mission tasks are issued, MATE dates are identical to last year, and
weekly Monday meetings):

\begin{itemize}
\item Pre-Nov 25th:
    \begin{itemize}
        \item Generic Software and Electronics Design
        \item Sponsorship Acquisition for Anticipated Parts
        \item Thruster, Camera Testing
        \item Pressure Chamber Design and Construction
    \end{itemize}
\item Nov 25:
    \begin{itemize}
        \item General Software and Electronics update
        \item Task-specific equipment design proposals assigned, equipment
testing begins
        \item Prop construction assigned
        \item Thruster Vectors discussed

    \end{itemize}
\item Dec 6: 
    \begin{itemize} 
        \item Final exams

    \end{itemize}
\item Dec 13:
    \begin{itemize} 
        \item Equipment testing update
         \item Prop construction update

    \end{itemize}
\item Dec 20:
    \begin{itemize} 
         \item Props finished
         \item Chassis basic design assigned (Basic design consists of a general
idea of what the thing will look like, what materials to use, what
size, etc)

    \end{itemize}
\item Dec 27:
    \begin{itemize} 
       \item Break for Christmas

    \end{itemize}
\item Jan 3:
    \begin{itemize} 
         \item Chassis basic design chosen, specific design assigned (specific
design includes dimensions, materials, part \#'s, etc)
         \item Equipment finished, final chosen
         \item Thruster vectors chosen
         \item Specific software design assigned (I'm assuming that software
will be an on-going process, like last year)
    \end{itemize}

\item Jan 10, 17, 24, 31
\item Feb 7, 14, 21:
    \begin{itemize} 
         \item Project updates
         \item Parts ordering
    \end{itemize}

\item Feb 28:
    \begin{itemize} 
         \item Product assembly assigned
    \end{itemize}

\item Mar 7:
    \begin{itemize} 
         \item Project updates
    \end{itemize}

\item Mar 14:
    \begin{itemize} 
     \item Final exams

    \end{itemize}
\item Mar 21, 28:
    \begin{itemize} 
         \item Project updates

    \end{itemize}
\item Apr 4:
    \begin{itemize} 
     \item UROV ready for practice runs (practice is on-going until
shipment)
    \end{itemize}

\item Apr 11, 18, 25

\item May 2, 9:
    \begin{itemize} 
         \item Project updates
    \end{itemize}

\item May 15th:
    \begin{itemize} 
         \item Regional qualifications
         \item Technical Report assigned
    \end{itemize}

\item May 27th:
    \begin{itemize} 
         \item Technical Report due, sent to MATE
    \end{itemize}

\item May 30th:
    \begin{itemize} 
         \item Project report
         \item Poster assigned
    \end{itemize}

\item June 6th:
    \begin{itemize} 
        \item Final exams
    \end{itemize}

\item June 13th:
    \begin{itemize} 
         \item Project update
         \item Poster finished
         \item Engineering evaluation practice
         \item UROV shipped
    \end{itemize}

\item June 20th:
    \begin{itemize} 
        \item Engineering evaluation practice
    \end{itemize}
\end{itemize}



\section{Management Plan}

Portland State aims to win the 2011 MATE underwater ROV competition in
June. Experience last year has convinced us of the need for certain changes.
First, the bar we need to reach to win is to not only accomplish all of the
tasks, but to do this in less time than the maximum allotted. The craft must be
fast, precise, and maneuverable. Two changes will achieve this: more powerful
thrusters and closed loop control of the motion. This will allow the operator
to quickly position the ROV, precisely as needed for the sensors and gripper,
instead of constantly fighting to establish and maintain position against cur-
rents and forces from the tether. Second, the team needs considerable practice
time. The team must be prepared to accomplish all the tasks with military-like
precision. The team must also practice for it's engineering evaluation. To en-
sure practice time before the competition this year's team is making changes.
One person is always designated the project manager. He or she has the final
say in any argument, hopefully this clear structure will prevent long arguments
from becoming distractions. A timeline has been written up and individual tasks
are being assigned. This structure gives team members the ability to work on
person-sized projects in a project-sized window. The hope here is that when
meetings are held, each individual can present work they have already done, as
opposed to long meetings where people argue about what they could do or want
to do.




\end{document}
