\documentclass{proposalnsf}
\usepackage{epsfig}

% NSF proposal generation template style file.
% based on latex stylefiles  written by Stefan Llewellyn Smith and
% Sarah Gille, with contributions from other collaborators.

\newcommand{\jas}{{\it J. Atmos. Sci.}}
\newcommand{\jpo}{{\it J. Phys. Oceanogr.}}
\newcommand{\JPO}{{\it J. Phys. Oceanogr.}}
\newcommand{\jfm}{{\it J. Fluid Mech.}}
\newcommand{\jgr}{{\it J. Geophys. Res.}}
\newcommand{\JGR}{{\it J. Geophys. Res.}}
\newcommand{\jmr}{{\it J. Mar. Res.}}
\newcommand{\arfm}{{\it Ann. Rev. Fluid Mech.}}
\newcommand{\dsr}{{\it Deep-Sea Res.}}
\newcommand{\dao}{{\it Dyn. Atmos. Oceans}}
\newcommand{\jam}{{\it Journal of Applied Meteorology}}
\newcommand{\phfl}{{\it Phys. Fluids}}
\newcommand{\phfla}{{\it Phys. Fluids A}}
\newcommand{\PhilTrans}{{\it Philosophical Transactions of the Royal Society, 
London}}
\newcommand{\gafd}{{\it Geophys. Astrophys. Fluid Dyn.}}
\newcommand{\gfd}{{\it Geophys. Fluid Dyn.}}
\newcommand{\PCE}   {{\it Physics and Chemistry of the Earth}}
\newcommand{\PRL}   {{\it Physical Review Letters}}

\newcommand{\ProgOc}{{\it Prog. Oceanography}}
\newcommand{\WHOITR}{Woods Hole Oceanographic Institution Technical Report, WHOI-}
\newcommand{\degrees}{$\!\!$\char23$\!$}
\DeclareFontFamily{OT1}{psyr}{}
\DeclareFontShape{OT1}{psyr}{m}{n}{<-> psyr}{}
\def\times{{\fontfamily{psyr}\selectfont\char180}}


\renewcommand{\refname}{\centerline{References cited}}

% this handles hanging indents for publications
\def\rrr#1\\{\par
\medskip\hbox{\vbox{\parindent=2em\hsize=6.12in
\hangindent=4em\hangafter=1#1}}}

\def\baselinestretch{1}

\begin{document}

\begin{center}
{\Large{\bf PDX-ROV 2010}}\\*[3mm]
{\bf 2010 Project Summary of NR Kobi} \\*[3mm]

Patrick Bledsoe, Gregory Haynes, Spencer Krum, Kristine Summerfield, Arthur Aldridge, Phil Witham, Keith Parker, Connor O'Conell, Jeff Doughty, Dr. Erik Sanchez

\end{center}

\noindent
{\bf PSU-ROV 2010 Summary in Brief}
{\it Total UROV cost: \$2496.06}
The Portland State University UROV team 2010 sent 5 students and 2 mentors to Hilo, Hawaii where the team competed against other UROV teams from around the world. The 2010 UROV, NR Kobi, received 70/300 mission points and 216/500 total points, 
ranking 17th out of 26 international teams.


\section{Introduction}
\noindent
{\bf MATE Competition:}
Each year the MATE center hosts an underwater remote operated vehicle competition. The competition is to accumulate the most points
in a simulated UROV mission taking place in a pool. There are two classes, one targeted at high schools and 
the other targeted at colleges and universities. PSU-UROV competed in the latter, the Explorer class. MATE 
publishes a list of missions and teams build UROVs specifically to complete these missions. The teams are under strict
 time constraints: 5 minutes to set up, 15 minutes in pool for missions, and a 5 minute clean up. The UROV pilots are not allowed to see the pool during the mission run; all 
control of the UROV must be done through the sensors the team has installed in the UROV. The UROV and all topside control equipment must be 
powered from a MATE-supplied DC power supply that provides up to 40 amps at 48V.

\ \\\ 
\noindent
{\bf Intellectual Merit:}
The design, construction, and operation of remote operated vehicles, and in particular underwater remote operated vehicles, is interesting
because these vehicles are a gateway to an ever expanding scientific frontier. Each new development in the technology behind these devices enable 
researchers to probe deeper into the ocean or further into caves. ROVs and UROVs are ubiquitous because 
of their utility; whether a team is looking at the bottom of a boat as a part of a safety check or probing a black smoker under the Pacific, 
they are doing it with an ROV because it's faster, cheaper, and safer to do so. UROVs can go places where manned submarines cannot, sometimes 
simply because the cost of deploying a manned submarine is too high.  
\ \\

\noindent
{\bf Broader Impacts:}
This project has been a learning experience for the team members and a chance to win some accolades for Portland State. Team members learned 
a variety of interrelated skills through this project. The realities of deadlines, interpersonal conflicts, and budgeting have been experienced, 
sometimes the hard way. There is a long list of skills that has been developed, ranging from soldering and machining components to deploying 
multi-processor closed-loop control systems. At the competition in Hawaii, the team will saw other team's solutions to the same problems and used the 
opportunity to network with leaders from both industry and academia. 
\renewcommand{\thepage} {B--\arabic{page}}


\section{Budget Summary}
\begin{tabular}[t]{lr}
\it Total UROV cost: &\$1144.43\\
\it Total Donated UROV cost in supplies and materials: &\$1351.63\\
\it Total Project cost in non-ROV materials: &\$1511.83\\
\it Total Donated Project cost including travel and board: &\$7851.63\\\hline
\it Total cost including donations: &\$9363.46\\
\end{tabular}


\noindent
\section{Mission}
 {\it Kristine wrote this part, I'm not sure reporting our challenges is the right idea... I also haven't had time to parse it yet}

Over 400 High Schools and Universities registered for the 2010 MATE International Remote Operated Vehicle Competition and underwent Regional Qualifications under the direction of
a judge certified by MATE.  Portland State University hosted the Regional Qualifications both this year and last year, and with the same judge.  As a mark of the PSU teams improvement
in project management, deadlines, dedication, and funding the team was commended on this years performance.  However, this years project suffered in several ways; in management, communication, and
practise.  

Last year the team put off most of the work on the construction of their ROV such that the design was lacking, this year there were alot of dedicated members that pushed for the completion
of the project to the best of their ability.  This dedication made the management of the project in terms of ideas robust but lacked a structured chain of command, one that suffered even more with the difficulties
of synchronizing schedules and without a singular project head to enforce deadlines.  In terms of management and communication it left the team floundering when decisions had to be made over design and practicality 
that ended with a more hodgepodge approach than anything.  
One difficulty was in the ever prevalent waterproofing of the arm servos, when the waterproofing technique was applied our team discovered that it 
was messy and not a very stable setup.  Despite the limitations of it the team continued to work with it instead of using the time to brainstorm another idea that could do the same job with less hassle.

Practical experience was another part of the project that was found lacking, without time to practise running the ROV several design flaws went undiscovered while the team spent most of their time
working on waterproofing.  For example the bilge pump motors used in the 2009 model were also used in the 2010 model, but without adequate practise and consideration of the new ROV
the ``slowness'' of such a design did not make itself apparent until it was too late to change it.  

Despite these setbacks Portland Sate University sent the comparatively small team of 5 students and 2 mentors to Hilo, Hawaii where the team showed off and competed with the operational ROV.  In total the team received 70/300 
mission points and 216/500 total points and ranking 17th out of 26 international teams, a vast improvement from last year.  The PSU UROV Team looks forward to the 2011 MATE Competition, where we can apply the hard-earned
lessons from the past two years and produce an ROV better than ever.
`
% reset page numbering to 1.  This is helpful, since the text can only
% be 15 pages, and reviewers will want to believe we've kept within
% those limits

\pagenumbering{arabic}
\renewcommand{\thepage} {D--\arabic{page}}

\section{Design}
\subsection{Mechanical}
The body of NR Kobi is a hollow stainless steel cylinder twenty-two cm  wide and forty cm long. Around this cylinder are girdles made of stainless steel which connect to skis or skids made of extruded aluminum. The skids are about sixy cm long. Between the girldes are horizontal 'wings' that provide structural rigidity and mounts for the four z direction motors. An injection molded clear plastic dome is mounted to the front of the steel cylinder. Four thrusters are mounted to the skids to control lateral movement. Each thruster is on a corner, pointed at forty five degrees away from perpindicular. This arrangement alows NR Kobi to propel itself forwards, backwards, left, and right, and to spin clockwise and counterclockwise. NR Kobi uses thrusters made from the motors from bilge pumps connected to plastic propellers from model boats.
\subsection{Electrical}
NR Kobi is powered from a standard MATE 48VDC/40A power supply. 48V is converted down to 12V by three syncronous regulator boards that were parted out of a telephone box. Two separate circuits are produced and separated by a shotkey diode. One circuit has a maximum current of 4 amps and is used to power the onboard electronics, the other has a maximum current of about 30A and is used to power the thrusters. The higher current in the latter circuit is obtained by connecting two voltage regulators in parallel and connecting them together in such a way that they share the load equally. Power for topside equipment such as laptops is provided by laptop batteries. The electronics circuit powers a 12V ATX power supply that was designed for automotive applications, this in turn provides power for the neocortex, NR Kobi's computer brain. NR Kobi is also equiped with a temperature sensor, the circut for which consists of a LM35 integrated circuit temperature sensor and a microcontroller to perform analog to digital conversion on the output pin of the LM35. Another(smaller) voltage regulator board takes 12V from the thruster circuit and converts it to 6V. This 6V is used for the temperature sensor and for DC servos on the arm. 
\subsection{Software}
The software behind NR Kobi enables two human pilots to see three live video streams and control the robot's position and it's robotic arm. There is one laptop topside and an x86 class processor and motherboard and two Atmel microcontrollers on-board the UROV. The laptop runs a program to control the UROV that is written in C++ and uses the Qt library to draw the gui and read from joysticks. Google Chrome is used to view the video stream, which is MJPEG over http. The software on the laptop communicates over ethernet with the software on the x86 using a JSON syntax because parsers for JSON were easy to obtain and implement. The x86 runs a program written in Python that sends/recieves commands and information with the control laptop and a program that grabs compressed JPEG's from usb webcams and pipes them over the network one after another over http in a process called MJPEG or Motion-JPEG. The Python running on the x86 uses PySerial to communicate with the two Atmel microcontrollers via a serial protocol over USB. The microcontrollers, which are programmed in C, accept commands over a very simple newline delimitted serial communication protocol and handle the analog to digital conversion for the sensors and output pulse width modulated signals for the motors. 



\newpage

\section{2011 Mission}
For the 2011 missions the big challenge will be the pressure and waterproofing of the ROV.  As the Competition will be held in the Neutral Buoyancy Laboratory of the Johnsonville Space Center
the ROVs will be required to reach a depth of ~40ft, and operate at a pressure the PSU Team currently has no way to test at.


\section{Design Projections for 2011}
Portland State aims to win the 2011 MATE underwater ROV competition in June. One of the biggest components of that goal is a world-class design. Central 
to this design are the best underwater thrusters available: 3 phase brushless motors running on 24 volts in an H-bridge configuration. The 24 volt power 
will be drawn from synchronous buck regulator boards that will be fed 48V DC. The regulator board(s) will be on the ROV meaning our tether will carry 48 
volt power instead of some lower voltage, reducing power lost as heat. We have a design that uses 6 thrusters and gives complete control over the ROV’s 
position in space to the operator; that is the ROV is capable of translation along the x, y, and z axes, and capable of rotation about the x, y, and z axes. 
Because we will have low power loss in the tether and in voltage level conversion, and because we are procuring the best available thrusters, we expect the 
ROV to have incredible speed, maneuverability, and thrust. The ROV will be controlled by a laptop computer running custom multi-platform software and will 
communicate with this computer using the RS232 serial protocol. The ROV will have an ARM microprocessor which will read and write to the controlling laptop, 
microcontrollers, and sensors. The sensors are one gyroscope and three monodirectional accelerometers, these sensors are externally powered and communicate 
with the ARM board via SPI. The microcontrollers are externally powered and communicate with the ARM board over RS232. They have 10bit analog to digital(ATD) 
conversion for mission specific sensing and about 8 pulse width modulated(PWM) signal pins which will be used to control the H bridges to control the thrusters. 
The ARM board will be running a closed-loop control system, listening to the gyroscope and activating the thrusters to hold the ROV in position. The 
controlling laptop will issue higher order commands in the form of vectors to the ARM board. The advantage to this scheme over a total control by the user 
scheme is it enables the ROV to hold position against a current or maintain a certain depth even as it takes on ballast.


\newpage
\pagenumbering{arabic}
\renewcommand{\thepage} {E--\arabic{page}}

\bibliography{draft}
\bibliographystyle{jponew}


\end{document}
