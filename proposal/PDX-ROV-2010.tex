\documentclass{article}
\begin{document}

\begin{center}
{\Large{\bf PSU-UROV 2010}}\\*[3mm]
{\bf 2010 Project Summary of NR Kobi} \\*[3mm]

Patrick Bledsoe, Gregory Haynes, Spencer Krum, Kristine Summerfield, Arthur Aldridge, Phil Witham, Keith Parker, Connor O'Conell, Jeff Doughty, Dr. Erik Sanchez

\end{center}

\noindent
{\bf PSU-UROV 2010 Summary in Brief}
{\it Total UROV cost: \$2496.06}
The 2010 Portland State University underwater remote operated vehicle(PSU-UROV) team 2010 sent 5 students and 2 mentors to Hilo, Hawaii where the team competed against other UROV teams from around the world. The 2010 UROV, NR Kobi, received 70/300 mission points and 216/500 total points, 
ranking 17th out of 26 international teams.


\section{Introduction}
\noindent
{\bf MATE Competition:}
Each year the Marine Advanced Technology Education(MATE) center hosts an underwater remote operated vehicle competition. The goal is to accumulate the most points
in a simulated UROV mission taking place in a pool. There are two competition classes, one targeted at high schools and 
the other targeted at colleges and universities. PSU-UROV competed in the latter, the Explorer class. MATE 
publishes a list of missions and teams build UROVs specifically to complete these missions. The teams are under strict
 time constraints: 5 minutes to set up, 15 minutes in pool for missions, and a 5 minute clean up. The UROV pilots are not allowed to see the pool during the mission run; all 
control of the UROV must be done through the sensors the team has installed in the UROV. The UROV and all topside control equipment must be 
powered from a MATE-supplied DC power supply that provides up to 40 amps at 48V.
\\

\noindent
{\bf Broader Impacts:}
The design, construction, and operation of remote operated vehicles, and in particular underwater remote operated vehicles, are interesting topics
because these vehicles are a gateway to an ever expanding scientific frontier. Each new development in the technology behind these devices enable 
researchers to probe deeper into the ocean or further into caves. ROVs and UROVs are ubiquitous because 
of their utility; whether a team is looking at the bottom of a boat as a part of a safety check or probing a black smoker under the Pacific, 
they are doing it with an ROV because it's faster, cheaper, and safer to do so. UROVs can go places where manned submarines cannot, sometimes 
simply because the cost of deploying a manned submarine is too high.  
\\

\noindent
{\bf Intellectual Merit:}
This project has been a learning experience for the team members and a chance to win some accolades for Portland State. Team members learned 
a variety of interrelated skills through this project. The team experienced the realities of meeting deadlines, resolving interpersonal conflicts, and budgeting. The team developed new skills, ranging from soldering and machining components to deploying 
multi-processor closed-loop control systems. At the competition in Hawaii, the team saw other team's solutions to the same problems and used the 
opportunity to network with leaders from both industry and academia. 



\section{Budget Summary}
\begin{tabular}[t]{lr}
\it Total UROV cost: &\$1,144.43\\
\it Total Donated UROV cost in supplies and materials: &\$1,351.63\\
\it Total Project cost in non-ROV materials: &\$1,511.83\\
\it Total Donated Project cost including travel and board: &\$7,851.63\\\hline
\it Total cost including donations: &\$9,363.46\\
\end{tabular}

\noindent
\section{Mission}
\noindent
{\bf Regionals:}
Portland State University played host to the MATE regional competition for the second year in a row. Two other Explorer class competitors drove to Portland for the reginal qualifications. To qualify, each ROV had to demonstrate controlled movement, maneuver into a cave, and pull a metal pin out from a socket. Portland State, Linn Benton Community College, and Clatsop Community College all qualified. Clatsop actually qualified with two ROVs.   
\\

\noindent
{\bf Hawaii:}
The MATE specified tasks were to sample a bacterial matt, to connect up a permanent hydrophone emplacement, to collect shrimp from a sea cave, and to test water temperature on a black smoker. The 300 mission points were distributed amongst these four tasks. In Hawaii, NR Kobi received 70 points for partially completing three of the tasks. On a positive side, NR Kobi received 70 points and the MIT team received only 65. 

\section{Team}
\noindent
{\bf NR Kobi:}
The name of the 2010 Portland State University underwater remote operated vehicle is NR Kobi. NR is US Navy shorthand for "underwater research vessel" and Kobi is the name of a dog. Our faculty sponsor, Dr. Erik S\'{a}nchez's dog was named Kobi. During the Spring, when the team was working hardest on the UROV, Erik had to put Kobi down and our team named the UROV Kobi in the dog's memory. NR Kobi is equipped with 8 thrusters, sensors, and a mechanical arm. 
\\

\noindent
{\bf Undergraduates:}
The 2010 team has five undergraduates. Patrick Bledsoe is a student from the Mechanical Engineering, he handled construction of the metal frame and most of the waterproofing. Spencer Krum is a student from Physics, he handled the electronics and assisted with software. Gregory Haynes is a student from Computer Science, he handled the programming and software. Arthur Aldridge is a student from Chemistry, he handled construction of the thrusters and assisted with metalwork. Kristine Summerfield is from Physics, she handled the documentation and technical report. 
\\

\noindent
{\bf Mentors and Faculty:}
Three mentors aided the undergraduates on the NR Kobi project. Keith Parker and Phil Witham helped with electronics and software. Jeff Doughty provided technical support and was a liaison to the Physics Department and Portland State as a whole. Dr. Erik S\'{a}nchez and Dr. Eric Bodegom provided support from the Physics Department and Dr. Bart Massey provided support from the Computer Science Department.

\section{Design}
\noindent
{\bf Mechanical:}
NR Kobi's on-board electronics are housed inside a hollow stainless steel
cylinder 22cm wide and 40cm long, with a plastic dome secured to the front,
and cables exiting from the back. Two sheets of stainless steel encircle
the cylinder. These girdle sheets are connected by "wings" welded between
the sheets. The wings are also stainless steel sheet, and have circles cut
into them that hold the vertical thrusters. The bottoms of the girdle
sheets are fastened to two 50cm pieces of extruded aluminum.
Four thrusters are mounted to the skids to control lateral movement.
Each thruster is on a corner, pointed forty five degrees away from
perpendicular.
This arrangement allows NR Kobi to propel itself forwards, backwards,
left, and right, and to spin clockwise and counterclockwise. NR Kobi
uses modified bilge pumps as thrusters, made by removing the impellers,
and fastening plastic propellers to the drive shaft of the pumps.
\\

\noindent
{\bf Electrical:}
NR Kobi is powered from a standard MATE 48VDC/40A power supply. 48V is converted down to 12V by three synchronous regulator boards that were parted out of a telephone box. Two separate circuits are produced and separated by a shotkey diode. One circuit has a maximum current of 4 amps and is used to power the on-board electronics, the other has a maximum current of about 30A and is used to power the thrusters. The higher current in the latter circuit is obtained by connecting two voltage regulators in parallel and connecting them together in such a way that they share the load equally. Power for topside equipment such as laptops is provided by laptop batteries. The electronics circuit powers a 12V ATX power supply that was designed for automotive applications, this in turn provides power for the neocortex, NR Kobi's computer brain. NR Kobi is also equiped with a temperature sensor, the circuit for which consists of a LM35 integrated circuit temperature sensor and a microcontroller to perform analog to digital conversion on the output pin of the LM35. Another(smaller) voltage regulator board takes 12V from the thruster circuit and converts it to 6V. This 6V is used for the temperature sensor and for DC servos on the arm. The thruster motors are controlled by feeding PWM signals from a microcontroller through a voltage divider and into the gate of a MOSFET. Each thruster is then hooked up in parallel with a decently fast diode to prevent back-EMF, and then that moiety is connected in series with the MOSFET and across the thruster power bus. This enables NR Kobi to turn on each of it's 8 thrusters independently, in only one direction, but with 8 bits of resolution. Power is also an issue; when more than two thrusters are running concurrently, the UROV has very little thrust. 
\\

\noindent
{\bf Software:}
The software behind NR Kobi enables two human pilots to see three live video streams and control the robot's position and it's robotic arm. There is one laptop topside and an x86 class processor and motherboard and two Atmel microcontrollers on-board the UROV. The laptop runs a program to control the UROV that is written in C++ and uses the Qt library to draw the gui and read from joysticks. Google Chrome is used to view the video stream, which is MJPEG over http. The software on the laptop communicates over ethernet with the software on the x86 using a JSON syntax because parsers for JSON were easy to obtain and implement. The x86 runs a program written in Python that sends/recieves commands and information with the control laptop and a program that grabs compressed JPEG's from usb webcams and pipes them over the network one after another over http in a process called MJPEG or Motion-JPEG. The Python running on the x86 uses PySerial to communicate with the two Atmel microcontrollers via a serial protocol over USB. The microcontrollers, which are programmed in C, accept commands over a very simple newline delimitted serial communication protocol and handle the analog to digital conversion for the sensors and output pulse width modulated signals for the motors. 


\section{2011 Mission}
For the 2011 missions the big challenge will be the pressure and waterproofing of the ROV.  As the Competition will be held in the Neutral Buoyancy Laboratory of the Johnsonville Space Center
the ROVs will be required to reach a depth of ~40ft, and operate at a pressure the PSU Team currently has no way to test at.


\section{Projections for 2011}

Portland State aims to win the 2011 MATE underwater ROV competition in June. Experience last year has convinced us of the need for certain changes. First, the bar we need to reach to win is to not only accomplish all of the tasks, but to do this in less time than the maximum alotted. The craft must be fast, precise, and maneuverable.  Two changes will achieve this: more powerful thrusters and closed loop control of the motion. This will allow the operator to quickly position the ROV, precisely as needed for the sensors and gripper, instead of constantly fighting to establish and maintain position against currents and forces from the tether. Second, the team needs considerable practice time. The team must be prepared to accomplish all the tasks with military-like precision. The team must also practice for it's engineering evaluation. To ensure practice time before the competition this year's team is making changes. One person is always designated the project manager. He or she has the final say in any argument, hopefully this clear structure will prevent long arguments from becoming distractions. A timeline has been writen up and individual tasks are being assigned. This structure gives team members the ability to work on person-sized projects in a project-sized window. The hope here is that when meetings are held, each individual can present work they have already done, as opposed to long meetings where people argue about what they could do or want to do.


\end{document}
