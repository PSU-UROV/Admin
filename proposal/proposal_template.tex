\documentclass{proposalnsf}
\usepackage{epsfig}

% NSF proposal generation template style file.
% based on latex stylefiles  written by Stefan Llewellyn Smith and
% Sarah Gille, with contributions from other collaborators.

\newcommand{\jas}{{\it J. Atmos. Sci.}}
\newcommand{\jpo}{{\it J. Phys. Oceanogr.}}
\newcommand{\JPO}{{\it J. Phys. Oceanogr.}}
\newcommand{\jfm}{{\it J. Fluid Mech.}}
\newcommand{\jgr}{{\it J. Geophys. Res.}}
\newcommand{\JGR}{{\it J. Geophys. Res.}}
\newcommand{\jmr}{{\it J. Mar. Res.}}
\newcommand{\arfm}{{\it Ann. Rev. Fluid Mech.}}
\newcommand{\dsr}{{\it Deep-Sea Res.}}
\newcommand{\dao}{{\it Dyn. Atmos. Oceans}}
\newcommand{\jam}{{\it Journal of Applied Meteorology}}
\newcommand{\phfl}{{\it Phys. Fluids}}
\newcommand{\phfla}{{\it Phys. Fluids A}}
\newcommand{\PhilTrans}{{\it Philosophical Transactions of the Royal Society, 
London}}
\newcommand{\gafd}{{\it Geophys. Astrophys. Fluid Dyn.}}
\newcommand{\gfd}{{\it Geophys. Fluid Dyn.}}
\newcommand{\PCE}   {{\it Physics and Chemistry of the Earth}}
\newcommand{\PRL}   {{\it Physical Review Letters}}

\newcommand{\ProgOc}{{\it Prog. Oceanography}}
\newcommand{\WHOITR}{Woods Hole Oceanographic Institution Technical Report, WHOI-}
\newcommand{\degrees}{$\!\!$\char23$\!$}
\DeclareFontFamily{OT1}{psyr}{}
\DeclareFontShape{OT1}{psyr}{m}{n}{<-> psyr}{}
\def\times{{\fontfamily{psyr}\selectfont\char180}}


\renewcommand{\refname}{\centerline{References cited}}

% this handles hanging indents for publications
\def\rrr#1\\{\par
\medskip\hbox{\vbox{\parindent=2em\hsize=6.12in
\hangindent=4em\hangafter=1#1}}}

\def\baselinestretch{1}

\begin{document}

\begin{center}
{\Large{\bf PSU-UROV 2011}}\\*[3mm]
{\bf Portland State Univeristy enters into the annual MATE international underwater remote operated vehicle competition} \\*[3mm]

Patrick Bledsoe, Gregory Haynes, Spencer O. Krum, Devin Quirozoliver, Kristine Summerfield,
 Allan Dunham, Dan Colish, Jeff Doughty, Dr. Erik Sanchez

\end{center}


This is a  proposal to fund a Portland State Univeristy team to compete in the Marine Advanced Technology Education Center (MATE)
 International Underwater Remote Operated Vehicle (UROV) competition. A team of undergraduates from Mechanical Engineering,
 Chemistry, Computer Science, and Physics are building a UROV to compete in June of 2011. The effort is sponsored by PSU, 
PSU Physics Dept, local buisnesses and companies, and private individuals. 2011 is the third consecutive year Portland State has sent 
a team to compete.  This year's team is strong in computer programming and experience from previous years. The combination of hard work 
from the team and support from you gives Portland State a real shot at winning this year's competition.  



\ \\
\noindent
{\bf Intellectual Merit:}
The design, construction, and operation of remote operated vehicles, and in particular underwater remote operated vehicles, is interesting
because these vehicles are a gateway to an ever expanding frontier. Each new development in the technology behind these devices enables 
researchers to probe deeper into the ocean or further into caves and even underwater rescue and repair. ROVs and UROVs are ubiquitous becasue 
of their utility, whether a team is looking at the bottom of a boat as a part of a safety check or probing a black smoker under the pacific, 
they are doing it with an ROV because it's faster, cheaper, and safer to do so. UROVs can go places where manned submarines cannot, sometimes 
simply because the cost of deploying a manned submarine is too high.  
\ \\

\noindent
{\bf Broader Impacts:}
This project is a learning experience for the team members and a chance to win some accolades for Portland State. Team members learn 
a variety of interrelated skills through this project. The realities of deadlines, interpersonal conflicts, and budgeting are learned, 
sometimes the hard way. There is a long list of skills that will be developed, ranging from soldering and machining components to deploying 
multi-processor closed-loop control systems. At the competiton, the team will see other team's solutions to the same problems and will be
 given the opportunity to network with leaders from both industry and academia. 
\renewcommand{\thepage} {B--\arabic{page}}

\newpage

% reset page numbering to 1.  This is helpful, since the text can only
% be 15 pages, and reviewers will want to believe we've kept within
% those limits

\pagenumbering{arabic}
\renewcommand{\thepage} {D--\arabic{page}}

\newpage

\centerline{\bf Results from Prior Competitions}

\noindent
{\bf PSU-ROV 2010}
{\it Total UROV cost: \$2496.06}

Out of over 400 applicants for the combind competition classes, Portland Sate Univeristy ROV team for the 2010 season qualified for the Internation Competition 
and then sent 5 students and 2 mentors to Hilo, Hawaii where the team competed successfully. The 2010 ROV recieved 70/300 mission points and 216/500 total points, 
ranking 17th out of 26 international teams.

\ \\
\noindent
{\bf PSU-ROV 2009}
{\it Total UROV cost: \$481.10}

The Portland State University ROV team for the 2009 season sent 3 students and 1 mentor to Boston, Mass. where the craft did not pass the safety inspection 
due to unforseen electrical diffaculties.  The craft recieved 0/300 mission points and an 80.67/500 total score, ranking 28th. 


\ \\
\noindent{\Large \bf PROJECT DESCRIPTION}

\section{Introduction}

Each year the MATE center hosts an underwater remote operated vehicle competition. There are two classes, one targeted at high schools and 
the other targeted at colleges and universities. The latter is called the Explorer class and is the class that PSU-UROV competes in. MATE 
publishes a list of missions and teams build UROVs specificially to complete these missions. MATE has not yet published the missions for 2011 
but some componets are standard requirements (i.e. an arm). The competition will take place in a pool for controlled conditions; in the 2011 
season this pool will be NASA's Neutral Boyancy Lab at the Johnsonville Space Center in Houston, TX. The UROV and all its control equipment 
must be powered from a supplied 48 volt DC power supply rated up to 40 amps provided by MATE. The team will be racing the clock, 
5 minutes to set up, 15 minutes in pool for missions, and a 5 minute clean up. The operators will not be allowed to see the pool durring the mission run; all 
control of the UROV must be done through the sensors the team has installed in the UROV. The UROV and all topside control equipment must be 
powered from a MATE-supplied DC power supply that in standard for all competators. 
\section{Mission}

MATE has not yet published the 2011 Mission objectives. Here is what is known:
\begin{itemize}
\item The 2010 mission objectives were published in late November, 2009
\item Explorer class qualifications will take place in May of 2011.
\item The mission will take place in June of 2011.
\item The Competition will be in Houston, in the pool NASA uses to train astronauts to work in zero g, ~40ft deep
\end{itemize}
It's really not a lot of information. The team is therefore working on things that aren't mission specific such as thrusters, gyroscopes, control loops, sponsorship, etc.
However, this years Challenge will be operating in the pressure at this ~40ft depth.


\section{Design}

Portland State aims to win the 2011 MATE underwater ROV competition in June. One of the biggest components of that goal is a world-class design. Experience last year has convinced us of the need for certain changes.  First, the bar we need to reach to win is to not only accomplish all of the tasks, but we must do this in less time than the maximum alotted.  The craft must be fast, precise, and maneuverable.  Two changes will achieve this: more powerful thrusters and closed loop control of the motion.  This will allow the operator to quickly position the ROV, precisely as needed for the sensors and gripper, instead of constantly fighting to establish and maintain position against currents and forces from the tether.  


Portland State aims to win the 2011 MATE underwater ROV competition in June. One of the biggest components of that goal is a world-class design. Central 
to this design are the best underwater thrusters available: 3 phase brushless motors running on 24 volts in an H-bridge configuration. The 24 volt power 
will be drawn from synchronous buck regulator boards that will be fed 48V DC. The regulator board(s) will be on the ROV meaning our tether will carry 48 
volt power instead of some lower voltage, reducing power lost as heat. We have a design that uses 6 thrusters and gives complete control over the ROV’s 
position in space to the operator; that is the ROV is capable of translation along the x, y, and z axes, and capable of rotation about the x, y, and z axes. 
Because we will have low power loss in the tether and in voltage level conversion, and because we are procuring the best available thrusters, we expect the 
ROV to have incredible speed, maneuverability, and thrust. The ROV will be controlled by a laptop computer running custom multi-platform software and will 
communicate with this computer using the RS232 serial protocol. The ROV will have an ARM microprocessor which will read and write to the controlling laptop, 
microcontrollers, and sensors. The sensors are one gyroscope and three monodirectional accelerometers, these sensors are externally powered and communicate 
with the ARM board via SPI. The microcontrollers are externally powered and communicate with the ARM board over RS232. They have 10bit analog to digital(ATD) 
conversion for mission specific sensing and about 8 pulse width modulated(PWM) signal pins which will be used to control the H bridges to control the thrusters. 
The ARM board will be running a closed-loop control system, listening to the gyroscope and activating the thrusters to hold the ROV in position. The 
controlling laptop will issue higher order commands in the form of vectors to the ARM board. The advantage to this scheme over a total control by the user 
scheme is it enables the ROV to hold position against a current or maintain a certain depth even as it takes on ballast.
\section{Yet Another Section}

\section{Time Line and Management Plan}

Meeting Agenda List (assumes that planned meetings start when the
mission tasks are issued, MATE dates are identical to last year, and
weekly Monday meetings):

Pre-Nov 25th:
     Generic Software and Electronics Design
     Sponsorship Acquisition for Anticipated Parts
     Thruster, Camera Testing
     Pressure Chamber Design and Construction

Nov 25:
     General Software and Electronics update
     Task-specific equipment design proposals assigned, equipment
testing begins
     Prop construction assigned
     Thruster Vectors discussed

Dec 6: Final exams

Dec 13:
     Equipment testing update
     Prop construction update

Dec 20:
     Props finished
     Chassis basic design assigned (Basic design consists of a general
idea of what the thing will look like, what materials to use, what
size, etc)

Dec 27:   Break for Christmas

Jan 3:
     Chassis basic design chosen, specific design assigned (specific
design includes dimensions, materials, part \#'s, etc)
     Equipment finished, final chosen
     Thruster vectors chosen
     Specific software design assigned (I'm assuming that software
will be an on-going process, like last year)

Jan 10, 17, 24, 31
Feb 7, 14, 21:
     Project updates
     Parts ordering

Feb 28:
     Product assembly assigned

Mar 7:
     Project updates

Mar 14: Final exams

Mar 21, 28:
     Project updates

Apr 4: UROV ready for practice runs (practice is on-going until
shipment)

Apr 11, 18, 25
May 2, 9:
     Project updates

May 15th:
     Regional qualifications
     Technical Report assigned

May 27th:
     Technical Report due, sent to MATE

May 30th:
     Project report
     Poster assigned

June 6th: Final exams

June 13th:
     Project update
     Poster finished
     Engineering evaluation practice
     UROV shipped

June 20th:
     Engineering evaluation practice


--------------------------------------------------------------------------------

This timeline is very aggressive once the mission tasks are given.
There's a little bit of slack given from mid Jan through Feb. I
haven't been working with you guys on software/controls/etc, so I
couldn't be very specific about the pre-Nov25th period.

Tell me what you think. Any suggested revisions?

\section{Summary:  Significance of proposed work}

\subsection{Intellectual Merit}

\subsection{Broader Impacts}



\newpage
\pagenumbering{arabic}
\renewcommand{\thepage} {E--\arabic{page}}

\bibliography{draft}
\bibliographystyle{jponew}

\newpage
\pagenumbering{arabic}
\renewcommand{\thepage} {G--\arabic{page}}
\noindent{\Large \bf BUDGET JUSTIFICATION}

\end{document}
